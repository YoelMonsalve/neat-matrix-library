\documentclass[11pt,letterpaper]{article}
%\usepackage[latin1]{inputenc}
\usepackage[english]{babel}
\usepackage{amsmath,amscd,amssymb}
\usepackage{amsfonts}
\usepackage{latexsym}
\usepackage{indentfirst}
\usepackage[dvips]{graphicx}
\usepackage{hyperref}
\usepackage[all]{xy}

%Para trabajar con pdftex:
%\usepackage[pdftex]{graphicx}
%\usepackage[pdftex]{color}

%Para trabajar con latex:
\usepackage[dvips]{graphics}
\usepackage[dvips]{color}

%Redefiniendo las funciones trigonom\'{e}tricas
\def\sen{\mbox{\rmfamily \upshape sen}\,}
\def\tg{\mbox{\rmfamily \upshape tg}\,}
\def\cotg{\mbox{\rmfamily \upshape cotg}\,}
\def\asen{\mbox{\rmfamily \upshape arc\,sen}\,}
\def\acos{\mbox{\rmfamily \upshape arc\,cos}\,}
\def\atan{\mbox{\rmfamily \upshape arc\,tan}\,}
\def\atg{\mbox{\rmfamily \upshape arc\,tg}\,}
\def\acot{\mbox{\rmfamily \upshape arc\,cot}\,}
\def\acotg{\mbox{\rmfamily \upshape arc\,cotg}\,}
\def\asec{\mbox{\rmfamily \upshape arc\,sec}\,}
\def\acosec{\mbox{\rmfamily \upshape arc\,cosec}\,}
\def\acsc{\mbox{\rmfamily \upshape arc\,csc}\,}

%Redefiniendo m\'{\i}n, m\'{a}x, l\'{\i}m,
\renewcommand{\min}[1]{\mbox{\rmfamily \upshape m\'{\i}n}\left\{  #1 \right\}}
\renewcommand{\max}[1]{\mbox{\rmfamily \upshape m\'{\i}n}\left\{  #1 \right\}}
\renewcommand{\lim}[1]{\mathop{\mbox{\rmfamily \upshape l\'{\i}m}}\limits_%
{\scriptsize #1}\quad}

%Definiendo operadores relacionados con n\'{u}meros complejos
\def\Ic{\mathbf{I}_{\mbox{\scriptsize \upshape \rmfamily C}}}
\def\IIc{\mathbf{II}_{\mbox{\scriptsize \upshape \rmfamily C}}}
\def\IIIc{\mathbf{III}_{\mbox{\scriptsize \upshape \rmfamily C}}}
\def\IVc{\mathbf{IV}_{\mbox{\scriptsize \upshape \rmfamily C}}}
\def\Re{\mbox{\rmfamily \upshape Re}\,}
\def\Im{\mbox{\rmfamily \upshape Im}\,}
\def\Arg{\mbox{\rmfamily \upshape Arg}\,}
\def\Res{\mathop{\mbox{\rmfamily \upshape Res}}}
\def\Log{\mbox{\rmfamily \upshape Log}}

%Redefiniendo algunas letras griegas
\def\alfa{\alpha}
\def\seta{\zeta}
\def\my{\mu}
\def\ny{\nu}
\def\ipsilon{\upsilon}
\def\ji{\chi}

%Redefiniendo algunos s\'{\i}mbolos matem\'{a}ticos comunes
\newcommand{\implica}{\ \Rightarrow\ }
\newcommand{\menos}{\setminus}
\newcommand{\binomial}{\choose}
\def\B{\mathfrak B}
\def\D{\mathfrak D}
\def\C{\mathbb C}
\def\E{\mathbb E}
\def\R{\mathbb R}
\def\Q{\mathbb Q}
\def\N{\mathbb N}
\def\L{\mathcal L}
\def\X{\mathbb X}
\def\Y{\mathbb Y}
\def\Z{\mathbb Z}
\def\I{\mathbf I}
\def\II{\mathbf II}
\def\III{\mathbf III}
\def\IV{\mathbf IV}
\def\({\left(}
\def\){\right)}
\def\[{\left[}
\def\]{\right]}
\def\iz|{\left|}
\def\de|{\right|}
\def\->{\rightarrow}
\def\<-{\leftarrow}
\def\d{\mbox{\rmfamily \upshape d}}
\def\e{\mbox{\rmfamily \upshape e}}

% Definiendo etiquetas de estilo (ejemplo, soluci'on, etc)
%\def\ejemplo{\noindent{\large\bfseries\sffamily\textcolor{blue}{Ejemplo.}}\quad}
%\def\solu{\noindent{\large\bfseries\sffamily\textcolor{blue}{Soluci\'on.}}\quad}

\def\example{\textsc{Example.\quad}}
\def\solution{\textbf{Solution.}}
\def\hsep{\rule{\textwidth}{0.5pt}\\}

% Definiendo s'imbolos de relaci'on y de conjuntos
\def\un{\mathop{\cup}}
\def\inter{\mathop{\cap}}

% Definiendo entornos de teorema, lema y corolario
\newtheorem{proposicion}{Proposici\'on}[section]
\newtheorem{teorema}{Teorema}[section]
\newtheorem{corolario}{Corolario}[teorema]
\newtheorem{lema}{Lema}[section]
\def\demo{{\sc\noindent DEMOSTRACI\'ON. }}

% Definiendo otros simbolos
\def\obs{{\sc\noindent OBSERVACION. }}
\newcommand{\obsN}[1]{{\sc\noindent OBSERVACI\'ON #1. }}
\def\cuadro{\scriptstyle\blacksquare}

\newcounter{ejemplo}\setcounter{ejemplo}0
\newcounter{ejercicio}\setcounter{ejercicio}0
\newcounter{definicion}\setcounter{definicion}0
\def\solu{\noindent{\bfseries\itshape {\color{blue}\large Soluci\'on.}}\quad}
\newcommand{\Ejemplo}{{\addtocounter{ejemplo}1\noindent\bfseries\sffamily\large
\colorbox[rgb]{1,1,.7}{\textcolor[rgb]{.2,.2,1}{Ejemplo
\arabic{ejemplo}. }}\quad }}
\newcommand{\Ejercicio}{{\addtocounter{ejercicio}1\noindent\scshape\Large {\textcolor[rgb]{.2,0,1}{\underline{Ejercicio No.
\arabic{ejercicio}.}}\quad}}}

%-----------------------------------------------------------------------------
%Definiendo otros aspectos del documento
\newcommand{\seccion}{\setcounter{equation}0\setcounter{ejercicio}0\setcounter{ejemplo}0\section}
\newcommand{\subseccion}{\setcounter{equation}0\setcounter{figure}0\subsection}
\def\theequation{\arabic{section}.\arabic{equation}}
\def\thefigure{\arabic{section}.\arabic{figure}}
\def\thetable{\arabic{section}.\arabic{table}}
\def\thepage{\arabic{page}}
\frenchspacing      %espaciado frances

\setlength\hoffset{-1in}     %sin separacion horizontal en la parte izquierda
\setlength\oddsidemargin{30mm}  %margen izquierdo
\setlength\textwidth{160mm} %ancho del texto

\setlength\voffset{-1in}        %sin separacion vertical en la parte superior
\setlength\topmargin{20mm}      %margen superior
\setlength\headheight{8mm}       %alto de la cabecera
\setlength\headsep{10mm}     %distancia de la cabecera al texto
\setlength\textheight{215mm} %alto del texto

%Dimensiones de la hoja tama~no carta:
%Ancho: 215.9 mm
%Alto: 279.4 mm
